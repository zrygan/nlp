\section{Declaration of AI Usage}

\subsection{Clarence Ivan Ang}

\textbf{AI Tool(s) Used:} ChatGPT

\textbf{Description of Use:} Understanding the process of machine learning and research

\textbf{Sample Prompts and Outputs:}
\begin{itemize}
    \item \textit{Example Prompt:} What are parameters in a model?
    \item \textit{Example Output:} In machine learning, parameters are the learnable numbers inside a model—the values the model adjusts during training so it can make accurate predictions...
\end{itemize}

\textbf{Extent of Use:} Moderate

\textbf{Reflection:} Using AI greatly contributed to my learning since I had no prior knowledge to machine learning at all. It helped me understand massively of aspects in machine learning to be able to understand and do this project.

\subsection{Clive Jarel Ang}

\textbf{AI Tool(s) Used:} GitHub Copilot, ChatGPT

\textbf{Description of Use:} Formatting and structuring the paper, organizing sections and subsections, and ensuring consistent LaTeX formatting throughout the document.

\textbf{Sample Prompts and Outputs:}
\begin{itemize}
    \item \textit{Example Prompt:} How should I structure the methodology section for a machine translation paper?
    \item \textit{Example Output:} A methodology section for machine translation typically includes: (1) Data preprocessing and tokenization, (2) Model architecture, (3) Training configuration, and (4) Translation strategies...
\end{itemize}

\textbf{Extent of Use:} Minimal

\textbf{Reflection:} Using AI for formatting help allowed me to focus on the content quality rather than spending excessive time on document structure. It streamlined the writing process and ensured consistency across all sections of the paper.

\subsection{Roan Cedric Campo}

\textbf{AI Tool(s) Used:} ChatGPT

\textbf{Description of Use:} Understanding documentation for Fairseq

\textbf{Sample Prompts and Outputs:}
\begin{itemize}
    \item \textit{Example Prompt:} How does Fairseq work?
    \item \textit{Example Output:} Fairseq is a sequence modeling toolkit from Facebook AI Research (Meta AI) designed for training and evaluating neural networks for tasks like machine translation, text generation, summarization, and language modeling. It’s powerful, modular, and widely used in research because it gives you a lot of control over models and training pipelines...
\end{itemize}

\textbf{Extent of Use:} Moderate

\textbf{Reflection:} Using AI greatly contributed to my learning since I had no prior knowledge to Fairseq at all. It helped me understand massively of aspects in the framework to be able to understand and implement the machine translation models for this project.

\subsection{Zhean Robby Ganituen}

\textbf{AI Tool(s) Used:} ChatGPT, Grok

\textbf{Description of Use:} Using AI to humanize and refine technical writing, making complex concepts more accessible and readable. Additionally used Grok for AI entertainment and exploring conversational AI capabilities.

\textbf{Sample Prompts and Outputs:}
\begin{itemize}
    \item \textit{Example Prompt:} Can you help me rephrase this sentence to sound more natural: "The model utilizes subword tokenization for morphological analysis"
    \item \textit{Example Output:} "The model uses subword tokenization to analyze morphological structure"
\end{itemize}

\textbf{Extent of Use:} Minimal

\textbf{Reflection:} Using AI to humanize technical writing helped me communicate complex ideas more clearly without losing accuracy. The entertainment aspect with Grok also gave me insights into how different AI systems approach conversation and problem-solving.
