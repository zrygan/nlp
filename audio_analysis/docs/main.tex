\documentclass{article}
\usepackage{tipa}
\usepackage{graphicx, cleveref}

\title{Audio Analysis}
\author{   
	Clarence Ang,
	Clive Ang,
	Roan Campo,
	Zhean Ganituen
}
\begin{document}
\maketitle

\section{Pronunciation of ``Halo-Halo''}

The Filipino dessert \emph{Halo-Halo} is often mispronounced, largely due to
differences between Tagalog and English orthography. The name is also
frequently confused with the common Tagalog expression \emph{halo-halo}
(``mixed together'').\footnote{English translation: \emph{mixed together}. For
	example, ``\emph{Pinag-halo-halo ang mga estudyante sa iba't ibang seksyon}'':
	``The students were \emph{mixed together} into different sections''.}

The literal pronunciation of the Filipino word ``halo-halo'' is a reduplication
of the lexeme ``halo'' as /\textipa{halo}/ or /\textipa{hal'o}/, with a glottal
stop occurring between the alveolar lateral approximant /\textipa{l}/ and the
close-mid back rounded vowel /\textipa{o}/. Another variant is the
reduplication of the phoneme ``halu'' as /\textipa{halu}/ or /\textipa{hal'u}/.

We discussed the variations of the pronunciations of \emph{Halo-Halo} and
\emph{halo-halo} and their analysis using PRAAT with the following voice actor
profile:

\begin{enumerate}
	\item \textbf{Actor:} Zhean Ganituen
	\item \textbf{Sex:} Male
	\item \textbf{Age range:} Young Adult (18--25)
	\item \textbf{Language:} English
	\item \textbf{Accent:} Philippine English
	\item \textbf{Speaking style:} Read speech
	\item \textbf{Emotion set:} Neutral
	\item \textbf{Audio format:} Mono channel recordings
\end{enumerate}

For brevity, we will use ``\emph{Halo-Halo}'' to refer to the dessert, and
``\emph{halo-halo}'' to refer to the adjectival or verbal expression.

\subsection{Mispronunciations of ``Halo-Halo''}

Figure~\ref{fig:halo} illustrates the spectrogram corresponding to each
mispronunciation of ``Halo-Halo''.

\paragraph{Reduplication of /\textipa{halo}/} The spectrogram in
\Cref{fig:halo} shows a continuous and repeating speech pattern.

\begin{figure}
	\centering
	\includegraphics[width=0.65\linewidth]{img/halo.png}
	\caption{Reduplication of /\textipa{halo}/}\label{fig:halo}
\end{figure}

\paragraph{Reduplication of /\textipa{hal'o}/} \Cref{fig:hal'o} shows repetitive speech pattern with
an abrupt stop in between each root word due to the glottal stop /\textipa{l'o}/. In fact, this is the
correct pronunciation of ``halo-halo''.

\begin{figure}
	\centering
	\includegraphics[width=0.65\linewidth]{img/hal_o.png}
	\caption{Reduplication of /\textipa{hal'o}/}\label{fig:hal'o}
\end{figure}

\paragraph{Reduplication of /\textipa{halu}/} \Cref{fig:halu} shows continuous and repeating
speech pattern.

\begin{figure}
	\centering
	\includegraphics[width=0.65\linewidth]{img/halu.png}
	\caption{Reduplication of /\textipa{halu}/}\label{fig:halu}
\end{figure}

\paragraph{Reduplication of /\textipa{hal'u}/} \Cref{fig:hal'u} shows repetitive speech pattern with
an abrupt stop in between each root word due to the glottal stop /\textipa{l'u}/.

\begin{figure}
	\centering
	\includegraphics[width=0.65\linewidth]{img/hal_u.png}
	\caption{Reduplication of /\textipa{hal'u}/}\label{fig:hal'u}
\end{figure}

We also see the differences between the /\textipa{u}/ and the /\textipa{o}/
sound in the spectrogram as /\textipa{u}/ tends to have a lower frequency than
/\textipa{o}/.

\begin{figure}
	\centering
	\includegraphics[width=0.65\linewidth]{img/u-o.png}
	\caption{Spectrogram of the /\textipa{u}/ and /\textipa{o}/ vowel sounds}\label{fig:u-o}
\end{figure}

It should be noted that the samples in
\cref{fig:halo,fig:hal'o,fig:halu,fig:hal'u} do not account for potential
allophonic changes in the syllable ``ha'' in the word ``Halo-Halo''. Here, we
simply assume that ``ha'' is pronounced identically in both duplications.

However,~\cite{KWF2015} indicates that ``Halo-Halo'' can be pronounced as
/\textipa{h'alo hal'o}/, with a glottal stop occurring between the glottal
fricative /\textipa{h}/ and the open front unrounded vowel /\textipa{a}/.
\Cref{fig:a-a'} illustrates both sounds in the spectrogram.

\begin{figure}
	\centering
	\includegraphics[width=0.65\linewidth]{img/a-a_.png}
	\caption{Spectrogram of the /\textipa{a}/ and /\textipa{'a}/ sounds}\label{fig:a-a'}
\end{figure}

The left spectrogram is the /\textipa{a}/ sound which we see is a continuous
sound that is slowly tapering off. While, the right is the /\textipa{'a}/ sound
which has an abrupt stop.

\subsection{Correct Pronunciation of ``Halo-Halo''}
Overall, we expect the spectrogram of ``Halo-Halo'' to display four distinct
phonetic sections. The first section ends abruptly due to a glottal stop. It is
followed by two syllables containing plain vowel sounds, and the final section
ends with a vowel accompanied by another glottal stop.

This pattern is observed in \cref{fig:correct}.

\begin{figure}
	\centering
	\includegraphics[width=0.65\linewidth]{img/correct.png}
	\caption{Spectrogram of the correct pronunciation of ``Halo-Halo''}\label{fig:correct}
\end{figure}

\subsection{Phonological Analysis of Mispronunciations}

The observed mispronunciations of ``Halo-Halo'' can be explained in terms of
\emph{first language (L1) interference}, particularly the influence of English
orthography and phonotactics on non-Tagalog
speakers~\cite{hansenedwards2008phonology}. English readers, when encountering
the sequence ``halo,'' often default to pronouncing the final \emph{``o''} as
/\textipa{u}/ rather than /\textipa{o}/. This reflects the English tendency to
interpret word-final ``o'' as a high back rounded vowel, rather than the
mid-back rounded vowel found in Tagalog.

Additionally, the absence of explicit diacritics in common spelling encourages
English speakers to treat ``Halo-Halo'' as a compound word with English stress
placement, leading to flattened intonation and the omission of Tagalog glottal
stops. The result is a pronunciation closer to /\textipa{halu-halu}/, which is
orthographically consistent with English but phonologically inconsistent with
Tagalog norms.

\subsection{Conclusion}

Variations in possible pronunciations of ``Halo-Halo'' arise from the
widespread use of \emph{common spelling} (writing without diacritics) in most
Philippine contexts. The word's pronunciation can be clearly inferred when
using its \emph{diacritic orthography}: ``h\'alo-Hal\`o'', which indicates the
stressed syllables and glottal stops.

\section{Audio Characteristics of Emotion Speech}

We recorded the phrase ``Your power is mine'' with the following voice actor
profile:

\begin{enumerate}
	\item \textbf{Actor:} Clarence Ang
	\item \textbf{Sex:} Male
	\item \textbf{Age range:} Young Adult (18--25)
	\item \textbf{Language:} English
	\item \textbf{Accent:} Philippine English
	\item \textbf{Speaking style:} Read speech with controlled emotional expressions
	\item \textbf{Emotion set:} Happy, Sad, Fear, Anger\footnote{One recording was produced for each emotion.}
	\item \textbf{Audio format:} Mono channel recordings
\end{enumerate}

The prosodic analysis of each emotional speech recording was conducted using
Praat. The characteristics of pitch and intensity for each emotion are
summarized in~\cref{tab:intensity,tab:pitch}:

\begin{table}[h]
	\centering
	\caption{Pitch characteristics of emotional recordings}
	\begin{tabular}{ll}
		\hline
		\textbf{Emotion} & \textbf{Pitch}                                      \\
		\hline
		Happy            & Relatively stable throughout the audio              \\
		Sad              & Relatively stable, with a much lower range          \\
		Anger            & Higher pitch compared to the other emotions         \\
		Fear             & Fluctuates (drops and rises) due to vocal shakiness \\
		\hline
	\end{tabular}\label{tab:pitch}
\end{table}

\begin{table}[h]
	\centering
	\caption{Intensity characteristics of emotional recordings}
	\begin{tabular}{ll}
		\hline
		\textbf{Emotion} & \textbf{Intensity}                               \\
		\hline
		Happy            & Can rise to high levels, then drop swiftly again \\
		Sad              & Generally low, with occasional slight increases  \\
		Anger            & Spikes sharply in waves, reaching high levels    \\
		Fear             & Unstable, rising and falling with uncertainty    \\
		\hline
	\end{tabular}\label{tab:intensity}
\end{table}

\subsection{Conclusion}
The emotional recordings exhibited distinct prosodic characteristics in terms
of pitch and intensity. For happy speech, the pitch remained relatively stable
while the intensity rose to high levels and dropped swiftly, reflecting
energetic expressiveness. Sad speech showed a much lower pitch range with
generally subdued intensity, though slight increases were observed at certain
points. In the case of anger, the pitch was markedly higher compared to the
other emotions, and the intensity spiked sharply in wave-like patterns,
producing a forceful delivery. Finally, fearful speech was characterized by
pitch fluctuations—dropping and rising due to vocal shakiness—while the
intensity was unstable, alternating between increases and decreases that
conveyed uncertainty.

\bibliography{refs}
\bibliographystyle{alpha}

\vfill
{\footnotesize
	\noindent\textbf{Declaration of AI Use:} The author used ChatGPT-5 to
	assist with language editing and improving the clarity of the manuscript.
	All content generated by the AI was carefully reviewed and edited by the
	authors. The authors take full responsibility for the accuracy and integrity
	of the final manuscript.
	\par}
\end{document}