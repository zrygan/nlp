\documentclass{article}
\usepackage{tipa}
\usepackage{graphicx, cleveref}

\title{Audio Analysis}
\author{   
    Clarence Ang,
    Clive Ang
    Roan Campo,
    Zhean Ganituen
}
\begin{document}
\maketitle

\section{Pronunciation of ``Halo-Halo''}

The Filipino dessert \emph{Halo-Halo} is often mispronounced, largely due to
differences between Tagalog and English orthography. The name is also
frequently confused with the common Tagalog expression \emph{halo-halo}
(``mixed together'').\footnote{English translation: \emph{mixed together}. For
    example, ``\emph{Pinag-halo-halo ang mga estudyante sa iba't ibang seksyon}'':
    ``The students were \emph{mixed together} into different sections''.}

The literal pronunciation of the Filipino word ``halo-halo'' is a reduplication
of the lexeme ``halo'' as /\textipa{halo}/ or /\textipa{hal'o}/, with a glottal
stop occurring between the alveolar lateral approximant [\textipa{l}] and the
close-mid back rounded vowel [\textipa{o}]. Another variant is the
reduplication of the phoneme ``halu'' as /\textipa{halu}/ or /\textipa{hal'u}/.

For brevity, we will use ``\emph{Halo-Halo}'' to refer to the dessert, and
``\emph{halo-halo}'' to refer to the adjectival or verbal expression.

\subsection{Mispronunciations of ``Halo-Halo''}

Figure~\ref{fig:halo} illustrates the spectrogram corresponding to each
mispronunciation of ``Halo-Halo''.

\paragraph{Reduplication of /\textipa{halo}/} The spectrogram in
\Cref{fig:halo} shows a continuous and repeating speech pattern.

\begin{figure}
    \centering
    \includegraphics[width=0.65\linewidth]{img/halo.png}
    \caption{Reduplication of /\textipa{halo}/}\label{fig:halo}
\end{figure}

\paragraph{Reduplication of /\textipa{hal'o}/} \Cref{fig:hal'o} shows repetitive speech pattern with
an abrupt stop in between each root word due to the glottal stop /\textipa{l'o}/. In fact, this is the
correct pronunciation of ``halo-halo''.

\begin{figure}
    \centering
    \includegraphics[width=0.65\linewidth]{img/hal_o.png}
    \caption{Reduplication of /\textipa{hal'o}/}\label{fig:hal'o}
\end{figure}

\paragraph{Reduplication of /\textipa{halu}/} \Cref{fig:halu} shows continuous and repeating
speech pattern.

\begin{figure}
    \centering
    \includegraphics[width=0.65\linewidth]{img/halu.png}
    \caption{Reduplication of /\textipa{halu}/}\label{fig:halu}
\end{figure}

\paragraph{Reduplication of /\textipa{hal'u}/} \Cref{fig:hal'u} shows repetitive speech pattern with
an abrupt stop in between each root word due to the glottal stop /\textipa{l'u}/.

\begin{figure}
    \centering
    \includegraphics[width=0.65\linewidth]{img/hal_u.png}
    \caption{Reduplication of /\textipa{hal'u}/}\label{fig:hal'u}
\end{figure}

We also see the differences between the /\textipa{u}/ and the /\textipa{o}/
sound in the spectrogram as /\textipa{u}/ tends to have a lower frequency than
/\textipa{o}/.

\begin{figure}
    \centering
    \includegraphics[width=0.65\linewidth]{img/u-o.png}
    \caption{Spectrogram of the /\textipa{u}/ and /\textipa{o}/ vowel sounds}\label{fig:u-o}
\end{figure}

It should be noted that the samples in
\cref{fig:halo,fig:hal'o,fig:halu,fig:hal'u} do not account for potential
allophonic changes in the syllable ``ha'' in the word ``Halo-Halo''. Here, we
simply assume that ``ha'' is pronounced identically in both duplications.

However,~\cite{KWF2015} indicates that ``Halo-Halo'' can be pronounced as
/\textipa{h'alo hal'o}/, with a glottal stop occurring between the glottal
fricative [\textipa{h}] and the open front unrounded vowel [\textipa{a}].
\Cref{fig:a-a'} illustrates both sounds in the spectrogram.

\begin{figure}
    \centering
    \includegraphics[width=0.65\linewidth]{img/a-a_.png}
    \caption{Spectrogram of the /\textipa{a}/ and /\textipa{'a}/ sounds}\label{fig:a-a'}
\end{figure}

The left spectrogram is the /\textipa{a}/ sound which we see is a continuous
sound that is slowly tapering off. While, the right is the /\textipa{'a}/ sound
which has an abrupt stop.

\subsection{Correct Pronunciation of ``Halo-Halo''}
Overall, we expect the spectrogram of ``Halo-Halo'' to display four distinct
phonetic sections. The first section ends abruptly due to a glottal stop. It is
followed by two syllables containing plain vowel sounds, and the final section
ends with a vowel accompanied by another glottal stop.

This pattern is observed in \cref{fig:correct}.

\begin{figure}
    \centering
    \includegraphics[width=0.65\linewidth]{img/correct.png}
    \caption{Spectrogram of the correct pronunciation of ``Halo-Halo''}\label{fig:correct}
\end{figure}

\subsection{Conclusion}

Variations in possible pronunciations of ``Halo-Halo'' arises from the
widespread use of \emph{common spelling} (writing without diacritics) in most
Philippine contexts. The word's pronunciation can be clearly inferred when
using its \emph{diacritic orthography}: ``h\'alo-Hal\`o'', which indicates the
stressed syllables and glottal stops.

\section{Audio Characteristics of Emotion Speech}

We recorded the phrase: ``Your power is mine''

\bibliography{refs}
\bibliographystyle{alpha}

\section*{Declaration of AI Use}
The author used ChatGPT-5 to assist with language editing and improving the clarity of the manuscript. All content
generated by the AI was carefully reviewed and edited by the authors. The authors take full responsibility for the
accuracy and integrity of the final manuscript.
\end{document}