% This must be in the first 5 lines to tell arXiv to use pdfLaTeX, which is strongly recommended.
\pdfoutput=1
% In particular, the hyperref package requires pdfLaTeX in order to break URLs across lines.

\documentclass[11pt]{article}

% Remove the "review" option to generate the final version.
\usepackage{ACL2023}

\usepackage{newtxtext,newtxmath}
\usepackage{latexsym}
\usepackage[T1]{fontenc}
\usepackage[utf8]{inputenc}
\usepackage{microtype}
\usepackage{inconsolata}

\usepackage{cleveref}
\usepackage{hyperref}

% Additional packages

% Define \blfootnote command for footnote without marker
\newcommand\blfootnote[1]{%
  \begingroup
  \renewcommand\thefootnote{}\footnote{#1}%
  \addtocounter{footnote}{-1}%
  \endgroup
}

\title{English Nativization Algorithm for Filipino Loan Words}
\setlength\titlebox{10cm}
\author{
  Clarence Ang \\
    De La Salle University \\
    \texttt{clarence\_ang@dlsu.edu.ph} \\\AND%
  Clive Ang \\
    De La Salle University \\
    \texttt{clive\_ang@dlsu.edu.ph} \\\AND%
  Roan Campo \\
    De La Salle University \\
    \texttt{roan\_campo@dlsu.edu.ph} \\\AND%
  Zhean Ganituen \\
    De La Salle University \\
    \texttt{zhean\_robby\_ganituen@dlsu.edu.ph} \\\AND%
  Nathaniel Oco \\
    De La Salle University \\
    \texttt{nathaniel.oco@dlsu.edu.ph} \\
}

\begin{document}
\maketitle
\begin{abstract}
We present a rule-based nativization algorithm for converting English loan words to Filipino orthography. The algorithm applies phonological transformation rules derived from the Komisyon sa Wikang Filipino (KWF) orthography guidelines. We evaluate the method on three datasets: generic drug names, drug brand names, and common English words used in Filipino. The algorithm achieves 88.94\% accuracy on generic drug names, 67.18\% on brand names, and 36\% on common English words. The results indicate that while rule-based approaches are effective for standardized terminology such as drug names, they face limitations when applied to words with irregular pronunciations.
\end{abstract}

\section{Introduction}

The Filipino language is one of the two official of the Philippines, the other
one being American English~\cite{Ethnologue}. Given that the country has two
official languages and are both widely used in multiple contexts, it is common
for Filipinos to use both languages simultaneously: borrow words through translation
or nativization, or code switching~\cite{Bautista_2004}.

To make things clear, translation is the process of converting a word from one
language to an equivalent word in another language. Here, equivalent means that
they share the same meaning. For example, the English word \textit{book} is
translated to \textit{aklat} in Filipino. While, nativization, more precisely
phonological nativization, is the process of adapting a foreign word to conform
to the phonological, morphological, and orthographic norms of the borrowing
language. For example, the English word \emph{computer} is nativized to
\emph{kompyuter} in Filipino. 

Although English is one of the official languages of the Philippines, it is not
the only language that influences Filipino. Other languages such as Spanish
(e.g., kutsara, kotse, diyos) and Chinese (e.g., kuya, ate, pansit) also
contribute loans words to Filipino. However, English loan words are more
prevalent in modern-day Filipino due its use in education, science, technology,
business, and other fields. Hence, this paper will purely focus on nativization
of English loan words to Filipino. And so, we leave the task of nativization
from other languages to Filipino as future work.

In particular, English is primarily used in the chemical and pharmaceutical
industries globally. The International Nonproprietary Names (INN) system,
established by the World Health Organization (WHO), provides a standardized
naming convention for pharmaceutical substances, which is predominantly based
on English~\cite{WHO_INN}. In the Philippines, the Food and Drug Administration
(FDA) adopts these INN names for drug registration and regulation and
pharmaceutical companies market and name their products using these
English-based names.

These standardization efforts allow for consistency and clarity in drug
identification and communication within the healthcare industry. We can see how
these standardization efforts have allowed work on natural language processing
(NLP) software, such as the United States FDA's Phonetic and Orthographic
Computer Analysis (POCA) Program which can confusible~\cite{FDA_POCA}. However,
this tools are primarily designed for English drug names and may not account for the
nativization patterns present in Filipino drug names. Thus, there is a need to
develop NLP tools that can effectively handle nativized Filipino drug names to
ensure accurate identification and communication within the Philippine context.

In this paper, we present a nativization algorithm for Filipino loan words,
specifically focusing on English drug names. The proposed method is evaluated
on three gold standard datasets. [... results]

The proposed method is a preliminary step that aims to contribute to the construction
of a more comprehensive NLP tool for Filipino drug names, which can aid in
reducing medication errors and improving patient safety in the Philippines.
\section{Algorithm}

% THIS IS CLARENCE'S ORIGINAL WRITE-UP
% INCORPORATE IF NEEDED LATER
% ===========================================
% In order to evaluate the nativization of drug names, we conducted preliminary research on standardized rules by the KWF on English loaned words. 
% Based on these translations, we conducted basic letter substitutions such as replacing ``ph" with ``f", ``z" with ``s", ``th" with ``t", ``ch" with ``k", ``v" with ``b", and 
% "x" with ``ks". Furthermore, there were other letter substitutions such as replacing ``c" with ``s" when followed by ``e", ``y", or ``i", and replacing ``c" with ``k" in other cases.
% This captures the soft and hard ``c" sounds in English. However, there are important limitations to note with the letter ``g". This letter has both a ``soft" and ``hard"
% pronunciation depending on the following vowel, similar to ``c". However, there are multiple special cases in which this algorithm is not followed. One such example
% is the word ``get", which has a hard ``g" sound despite being followed by ``e". Due to these exceptions, there are limitations to drug names that contain the letter ``g" within them.

% Aside from these basic letter substitutions, there are other special exceptions and rules that were applied. For example, any drug name that has ``ee" in it's spelling will be
% substituted with ``i". The same occurrence applies to ``oo", which will be replaced with the letter ``u". However other characters that have duplicated letters in a row are substituted with just 
% one letter instead. Additionally, if a drug name ends with the letter ``e", it is highly noticeable that the ``e" is silent and removed in pronunciation. Therefore, any drug name that ends with ``e" 
% will have the final ``e" removed. Moreoever, there are multiple drug names that end with the following: ``-ide", ``-ate", and ``-one", which are typically pronounced as ``ayd", ``eyt", and ``own" 
% respectively. To account for these common endings, we have replaced the endings accordingly. There are also a considerable amount of other spellings in drug names such as ``ein" that gets 
% substituted with ``in" and ``thi" that may follow with a ``a" and ``o" with ``ti[a/o]" depending on the following letter.

% Lastly, one of the biggest limitations that we faced whilst encountering the drug names were the various difference of pronunciations between ``y", ``i", and ``ay". Throughout different drug names,
% there are multiple pronunciations that may involve these letters. For example, the drug name ``diphenhydramine" has the first ``i" as a ``ay" sound, compared to the second ``i" which has a ``ee" sounnd.
% These multiple different types of of pronunciations made it difficult to standardize a single rule based on graphemes on how to nativize these drug names. There are other occurences in which two words such as 
% ``triamcinolone" and ``hydrochlorothiazide" in which possess both ``ria" and ``thia" respectively. In these cases, ``ria" is pronounced as ``ree-ea" while ``thia'' is pronounced as ``thi-a". Due to these
% cases, there occurs a limitation on being able to diagnose the correct letter substitution for these types of pronunciations.
% ===========================================
% THIS IS CLARENCE'S ORIGINAL WRITE-UP
% INCORPORATE IF NEEDED LATER

The regulating body for the Filipino language of the Philippine government is
the Komisyon sa Wikang Filipino (KWF) provides a manual on the orthography of
the language~\cite{KWF_MMP} which includes guidelines on nativization of loan
words.

In particular, the KWF provides a list of substitutions for common letter
combinations and sounds found in English loan words to their Filipino
equivalents. Based on these guidelines, we developed a rule-based algorithm to
nativize English drug names to Filipino.

We enumerate the types of rules required for the nativization algorithm below.

\subsection{Monograph Substitutions}

These are basic letter substitutions where a single letter in English is
replaced with a single letter in Filipino. Since both languages use the Latin
script, many letters have direct equivalents while some being equivalent with
itself. For example, the letter ``a'' (\textbf{a}bstact) in English is pronounced
the same way as the letter ``a'' (\textbf{a}bstrak) in Filipino. While the letter
``v'' (\textbf{v}olleyball) does not exist in Filipino phonetics and is replaced
with the letter ``b'' (\textbf{b}alibol) which has a similar sound. Furthermore,
other letters such as ``x'' (inde\textbf{x}, \textbf{x}-ray) and ``q'' (\textbf{q}ueen) do not
a single letter mapping, they are instead replaced with ``ks'' (inde\textbf{ks},
\textbf{eks}-ray) and ``kuw'' (\textbf{kuw}in).

In the table below, we summarize the common monograph substitutions used in our
nativization algorithm.

\begin{table}[h]
\centering
\caption{Monograph Phoneme Mappings}
\begin{tabular}{|c|c|c|c|}
\hline
\textbf{Source} & \textbf{Target} & \textbf{Source} & \textbf{Target} \\
\hline
b & b & p & p \\
k & k & r & r \\
d & d & s & s \\
g & g & t & t \\
h & h & w & w \\
l & l & y & y \\
m & m & v & b \\
n & n & f & p \\
\hline
j & dy & z & s \\
q & k & x & ks \\
\hline
\end{tabular}
\end{table}

\subsection{Digraphs}

English has several letter combinations or digraphs that represent specific
sounds. 

We again use the KWF manual to identify common digraphs in English and their
Filipino equivalents. For example, the digraph ``ch'' (\textbf{ch}irp) is
mapped to ``ts'' (\textbf{ts}irp) in Filipino. Another example is the digraph
``sh'' (\textbf{sh}ip, squi\textbf{sh}) which is replaced with ``sy'' (\textbf{sy}ip) or ``s'' (\textbf{iskuwis}) in
Filipino. A special diragph is the ``ng'' which is the only digraph that is a
letter in the Filipino alphabet.

Furthermore, English also uses a few vowel digraphs that are mapped to Filipino
equivalents. For example, the digraph ``oo'' (\textbf{oo}ze) is replaced with
``u'' (\textbf{u}z) in Filipino. Another example is the digraph ``ea''
(\textbf{ea}sy) which is mapped to ``i'' (\textbf{i}si) in Filipino.

In the table below, we summarize the common digraph substitutions used in our
nativization algorithm.


\begin{table}[h]
\centering
\caption{Digraph Phoneme Mappings}
\begin{tabular}{|c|c|}
\hline
\textbf{Source} & \textbf{Target} \\
\hline
ph & p \\
th & t \\
ch & ts \\
sh & sy \\
ng & ng \\
ny & ny \\
qu & kw \\
ñ & ny \\
\hline
\end{tabular}
\end{table}

\subsection{Vowel Unpredictability}

Lastly, one of the biggest limitations that we faced whilst encountering the
drug names were the various difference of pronunciations between ``y", ``i",
and ``ay". Throughout different drug names, there are multiple pronunciations
that may involve these letters. For example, the drug name ``diphenhydramine"
has the first ``i" as a ``ay" sound, compared to the second ``i" which has a
``ee" sounnd. These multiple different types of of pronunciations made it
difficult to standardize a single rule based on graphemes on how to nativize
these drug names. There are other occurences in which two words such as
``triamcinolone" and ``hydrochlorothiazide" in which possess both ``ria" and
``thia" respectively. In these cases, ``ria" is pronounced as ``ree-ea" while
``thia'' is pronounced as ``thi-a". Due to these cases, there occurs a
limitation on being able to diagnose the correct letter substitution for these
types of pronunciations.
\section{Experimental Setup}

To evaluate the nativization algorithm, three datasets of English words with
their corresponding Filipino nativized forms were constructed.

\subsection{Datasets}

The first dataset consists of 199 generic drug names. These are International
Nonproprietary Names (INN) of pharmaceutical substances commonly available in
the Philippines. The names were compiled from publicly available pharmaceutical
references and represent drugs across various therapeutic categories.

The second dataset consists of 262 drug brand names registered with the
Philippines Food and Drug
Administration.\footnote{\url{https://verification.fda.gov.ph/}} These include
proprietary names used by pharmaceutical companies in the Philippine market.
Unlike generic names which follow INN conventions, brand names exhibit more
varied orthographic patterns.

The third dataset consists of 100 common English words frequently used in
Filipino conversations. These words were collected from posts on the r/Tagalog
subreddit,\footnote{\url{https://www.reddit.com/r/Tagalog/}} representing
informal usage of English loan words in everyday Filipino.

Each dataset is stored as comma-separated values with the English word and its
corresponding Filipino nativization. Table~\ref{tab:examples} shows sample
entries from each dataset.

\begin{table}[h]
\centering
\begin{tabular}{lll}
\hline
\textbf{Dataset} & \textbf{English} & \textbf{Filipino} \\
\hline
Generic drugs & acetaminophen & asetaminofen \\
              & amoxicillin & amoksisilin \\
              & azithromycin & asitromaysin \\
\hline
Brand names   & abilify & abilifay \\
              & votrient & botriyent \\
              & vitaplus & baytaplus \\
\hline
Common words  & sorry & sori \\
              & download & dawnlowd \\
              & screenshot & skrinshat \\
\hline
\end{tabular}
\caption{Sample nativization pairs from each dataset.}
\label{tab:examples}
\end{table}

\subsection{Gold Standard Construction}

For each word in the datasets, the corresponding Filipino nativized form was
manually created by the authors, who are native Filipino speakers, following
the orthographic guidelines from the KWF Manwal sa Masinop na
Pagsulat~\cite{KWF_MMP}.

\subsection{Evaluation Metric}

The algorithm's output was compared against the gold standard nativizations
using exact string matching. Accuracy was computed as the proportion of words
where the algorithm's output exactly matched the expected nativized form. An
analysis of common error patterns is presented in the Results section.

\section{Results and Discussion}

We evaluated the algorithm on the three datasets described in the experimental
setup: 199 generic drug names, 262 drug brand names, and 100 common English
words used in Filipino. The performance was measured as accuracy, computed as
the proportion of words for which the algorithm's output exactly matched the
manually curated nativized form.

On the generic drug names, the system achieved an accuracy of 88.94\%,
correctly producing nativized forms such as \textit{acetaminophen}
$\rightarrow$ \textit{asetaminofen} and \textit{azithromycin} $\rightarrow$
	extit{asitromaysin}. Accuracy decreased to 67.18\% on brand names and 36\%
on common English words. This pattern reflects the greater orthographic and
phonological variability of proprietary names and informal borrowings compared
with standardized International Nonproprietary Names.

Many of the remaining errors involve the treatment of ``y'', ``i'', and the
diphthong ``ay'', particularly in cases where multiple plausible Filipino
realizations exist. Words containing sequences such as ``ria'' or ``thia" may
be pronounced and nativized in more than one way, and the current set of
rules does not capture all of these possibilities. These observations
underscore the difficulty of modeling fine-grained pronunciation and
orthographic patterns through fixed rules alone and motivate future work that
incorporates phonetic information or data-driven methods to improve coverage
on brand names and common words.
\section{Conclusion}

The findings of this study highlight both the strengths and limitations of the proposed nativization 
algorithm. While the system demonstrated strong performance when applied to drug names—its primary
target domain—the significant drop in accuracy across brand names and common words underscores the 
constraints of a predominantly rule-based approach. In particular, the algorithm’s inconsistent 
handling of the ``y", ``ay", and ``i" variations reflects the inherent difficulty of modeling nuanced 
phonological and orthographic behaviors using fixed rules alone. These limitations indicate that the 
algorithm, in its current form, lacks the flexibility needed to fully generalize beyond the structured
patterns present in the initial dataset.

Despite these challenges, the tool effectively fulfills its intended purpose of generating nativized 
forms of drug names in Filipino, establishing a solid foundation for future enhancements. Moving forward, 
incorporating data-driven methods—such as phonetic modeling, statistical approaches, or machine learning
techniques—may offer more robust handling of vowel-glide patterns and irregularities across broader word 
categories. Such improvements would not only address the current constraints but also expand the algorithm’s 
versatility, ensuring more consistent and accurate nativization across diverse linguistic inputs.


\section*{Limitations}

The proposed algorithm has several limitations that should be considered when interpreting the results and applying the method to other contexts.

\paragraph{Grapheme-based approach.} The algorithm operates purely on orthographic representations and does not utilize phonetic information. English orthography lacks a consistent one-to-one correspondence between graphemes and phonemes. For instance, the letter ``i'' in ``diphenhydramine'' represents two distinct sounds: /a\textsc{i}/ (as in ``eye'') for the first occurrence and /\textsc{i}/ (as in ``bit'') for the second. Without access to phonetic transcriptions or pronunciation dictionaries, the algorithm cannot reliably distinguish such cases, leading to incorrect nativizations.

\paragraph{Domain specificity.} The transformation rules were derived primarily from patterns observed in pharmaceutical terminology. Generic drug names follow relatively standardized naming conventions based on International Nonproprietary Names (INN), which exhibit consistent phonological patterns. The lower accuracy on common English words (36\%) indicates that these rules do not transfer well to the broader lexicon, where historical borrowing patterns, regional variations, and irregular pronunciations introduce inconsistencies that rule-based methods cannot capture.

\paragraph{Gold standard construction.} The reference nativizations were created by the authors, who are native Filipino speakers, following the KWF orthography guidelines. However, no inter-annotator agreement was computed, and nativization can be inherently subjective. Different speakers may produce varying nativized forms for the same source word depending on dialect, education, and individual pronunciation habits.

\paragraph{Scope of source languages.} This work addresses only English-to-Filipino nativization. The Filipino language has historically borrowed from multiple languages, including Spanish (e.g., ``kutsara'' from \textit{cuchara}) and Hokkien Chinese (e.g., ``pansit'' from \textit{pi\^{a}n-sit}). The nativization patterns for these source languages differ from English and are not covered by the proposed algorithm.

\section*{Ethics Statement}

The datasets used in this study were compiled from publicly available sources. The generic drug names consist of International Nonproprietary Names (INN), which are standardized designations published by the World Health Organization. The brand names were obtained from the Philippines Food and Drug Administration's public verification database. The common English words were collected from publicly accessible posts on the r/Tagalog subreddit. No private, proprietary, or personally identifiable information was collected or used in this research.

This algorithm is developed for research purposes and is not intended for direct clinical or pharmaceutical application. We acknowledge that nativization of drug names in healthcare settings carries inherent risks. Errors in nativization could result in phonetically similar outputs for distinct drug names, potentially contributing to confusion and medication errors. Therefore, any deployment of this or similar tools in clinical contexts should incorporate verification by qualified healthcare professionals and should not replace established drug name safety protocols.

\section*{Acknowledgments}
This work represents a preliminary work as part of an ongoing undergraduate thesis research project by Zhean Ganituen and others, advised by Nathaniel Oco, on computing confusible drug names in the Philippine setting.

% Entries for the entire Anthology, followed by custom entries
\bibliography{anthology,custom}
\bibliographystyle{acl_natbib}

% \appendix

% \section{Example Appendix}
% \label{sec:appendix}

% This is a section in the appendix.

\end{document}
