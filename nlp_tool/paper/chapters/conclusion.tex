\section{Conclusion}

The findings of this study highlight both the strengths and limitations of the proposed nativization 
algorithm. While the system demonstrated strong performance when applied to drug names—its primary
target domain—the significant drop in accuracy across brand names and common words underscores the 
constraints of a predominantly rule-based approach. In particular, the algorithm’s inconsistent 
handling of the ``y", ``ay", and ``i" variations reflects the inherent difficulty of modeling nuanced 
phonological and orthographic behaviors using fixed rules alone. These limitations indicate that the 
algorithm, in its current form, lacks the flexibility needed to fully generalize beyond the structured
patterns present in the initial dataset.

Despite these challenges, the tool effectively fulfills its intended purpose of generating nativized 
forms of drug names in Filipino, establishing a solid foundation for future enhancements. Moving forward, 
incorporating data-driven methods—such as phonetic modeling, statistical approaches, or machine learning
techniques—may offer more robust handling of vowel-glide patterns and irregularities across broader word 
categories. Such improvements would not only address the current constraints but also expand the algorithm’s 
versatility, ensuring more consistent and accurate nativization across diverse linguistic inputs.