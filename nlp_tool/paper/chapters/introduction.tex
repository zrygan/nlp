\section{Introduction}

The Filipino language is one of the two official of the Philippines, the other
one being American English~\cite{Ethnologue}. Given that the country has two
official languages and are both widely used in multiple contexts, it is common
for Filipinos to use both languages simultaneously: borrow words through translation
or nativization, or code switching~\cite{Bautista_2004}.

To make things clear, translation is the process of converting a word from one
language to an equivalent word in another language. Here, equivalent means that
they share the same meaning. For example, the English word \textit{book} is
translated to \textit{aklat} in Filipino. While, nativization, more precisely
phonological nativization, is the process of adapting a foreign word to conform
to the phonological, morphological, and orthographic norms of the borrowing
language. For example, the English word \emph{computer} is nativized to
\emph{kompyuter} in Filipino. 

Although English is one of the official languages of the Philippines, it is not
the only language that influences Filipino. Other languages such as Spanish
(e.g., kutsara, kotse, diyos) and Chinese (e.g., kuya, ate, pansit) also
contribute loans words to Filipino. However, English loan words are more
prevalent in modern-day Filipino due its use in education, science, technology,
business, and other fields. Hence, this paper will purely focus on nativization
of English loan words to Filipino. And so, we leave the task of nativization
from other languages to Filipino as future work.

In particular, English is primarily used in the chemical and pharmaceutical
industries globally. The International Nonproprietary Names (INN) system,
established by the World Health Organization (WHO), provides a standardized
naming convention for pharmaceutical substances, which is predominantly based
on English~\cite{WHO_INN}. In the Philippines, the Food and Drug Administration
(FDA) adopts these INN names for drug registration and regulation and
pharmaceutical companies market and name their products using these
English-based names.

These standardization efforts allow for consistency and clarity in drug
identification and communication within the healthcare industry. We can see how
these standardization efforts have allowed work on natural language processing
(NLP) software, such as the United States FDA's Phonetic and Orthographic
Computer Analysis (POCA) Program which can confusible~\cite{FDA_POCA}. However,
this tools are primarily designed for English drug names and may not account for the
nativization patterns present in Filipino drug names. Thus, there is a need to
develop NLP tools that can effectively handle nativized Filipino drug names to
ensure accurate identification and communication within the Philippine context.

In this paper, we present a nativization algorithm for Filipino loan words,
specifically focusing on English drug names. The proposed method is evaluated
on three gold standard datasets. [... results]

The proposed method is a preliminary step that aims to contribute to the construction
of a more comprehensive NLP tool for Filipino drug names, which can aid in
reducing medication errors and improving patient safety in the Philippines.