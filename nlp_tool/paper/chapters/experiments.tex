\section{Experimental Setup}

To evaluate the nativization algorithm, three datasets of English words with
their corresponding Filipino nativized forms were constructed.

\subsection{Datasets}

The first dataset consists of 199 generic drug names. These are International
Nonproprietary Names (INN) of pharmaceutical substances commonly available in
the Philippines. The names were compiled from publicly available pharmaceutical
references and represent drugs across various therapeutic categories.

The second dataset consists of 262 drug brand names registered with the
Philippines Food and Drug
Administration.\footnote{\url{https://verification.fda.gov.ph/}} These include
proprietary names used by pharmaceutical companies in the Philippine market.
Unlike generic names which follow INN conventions, brand names exhibit more
varied orthographic patterns.

The third dataset consists of 100 common English words frequently used in
Filipino conversations. These words were collected from posts on the r/Tagalog
subreddit,\footnote{\url{https://www.reddit.com/r/Tagalog/}} representing
informal usage of English loan words in everyday Filipino.

Each dataset is stored as comma-separated values with the English word and its
corresponding Filipino nativization. Table~\ref{tab:examples} shows sample
entries from each dataset.

\begin{table}[h]
\centering
\begin{tabular}{lll}
\hline
\textbf{Dataset} & \textbf{English} & \textbf{Filipino} \\
\hline
Generic drugs & acetaminophen & asetaminofen \\
              & amoxicillin & amoksisilin \\
              & azithromycin & asitromaysin \\
\hline
Brand names   & abilify & abilifay \\
              & votrient & botriyent \\
              & vitaplus & baytaplus \\
\hline
Common words  & sorry & sori \\
              & download & dawnlowd \\
              & screenshot & skrinshat \\
\hline
\end{tabular}
\caption{Sample nativization pairs from each dataset.}
\label{tab:examples}
\end{table}

\subsection{Gold Standard Construction}

For each word in the datasets, the corresponding Filipino nativized form was
manually created by the authors, who are native Filipino speakers, following
the orthographic guidelines from the KWF Manwal sa Masinop na
Pagsulat~\cite{KWF_MMP}.

\subsection{Evaluation Metric}

The algorithm's output was compared against the gold standard nativizations
using exact string matching. Accuracy was computed as the proportion of words
where the algorithm's output exactly matched the expected nativized form. An
analysis of common error patterns is presented in the Results section.
