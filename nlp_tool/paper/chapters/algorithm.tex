\section{Algorithm}


In order to evaluate the nativization of drug names, we conducted preliminary research on standardized rules by the KWF on English loaned words. 
Based on these translations, we conducted basic letter substitutions such as replacing ``ph" with ``f", ``z" with ``s", ``th" with ``t", ``ch" with ``k", ``v" with ``b", and 
"x" with ``ks". Furthermore, there were other letter substitutions such as replacing ``c" with ``s" when followed by ``e", ``y", or ``i", and replacing ``c" with ``k" in other cases.
This captures the soft and hard ``c" sounds in English. However, there are important limitations to note with the letter ``g". This letter has both a ``soft" and ``hard"
pronunciation depending on the following vowel, similar to ``c". However, there are multiple special cases in which this algorithm is not followed. One such example
is the word ``get", which has a hard ``g" sound despite being followed by ``e". Due to these exceptions, there are limitations to drug names that contain the letter ``g" within them.

Aside from these basic letter substitutions, there are other special exceptions and rules that were applied. For example, any drug name that has ``ee" in it's spelling will be
substituted with ``i". The same occurrence applies to ``oo", which will be replaced with the letter ``u". However other characters that have duplicated letters in a row are substituted with just 
one letter instead. Additionally, if a drug name ends with the letter ``e", it is highly noticeable that the ``e" is silent and removed in pronunciation. Therefore, any drug name that ends with ``e" 
will have the final ``e" removed. Moreoever, there are multiple drug names that end with the following: ``-ide", ``-ate", and ``-one", which are typically pronounced as ``ayd", ``eyt", and ``own" 
respectively. To account for these common endings, we have replaced the endings accordingly. There are also a considerable amount of other spellings in drug names such as ``ein" that gets 
substituted with ``in" and ``thi" that may follow with a ``a" and ``o" with ``ti[a/o]" depending on the following letter.

Lastly, one of the biggest limitations that we faced whilst encountering the drug names were the various difference of pronunciations between ``y", ``i", and ``ay". Throughout different drug names,
there are multiple pronunciations that may involve these letters. For example, the drug name ``diphenhydramine" has the first ``i" as a ``ay" sound, compared to the second ``i" which has a ``ee" sounnd.
These multiple different types of of pronunciations made it difficult to standardize a single rule based on graphemes on how to nativize these drug names. There are other occurences in which two words such as 
``triamcinolone" and ``hydrochlorothiazide" in which possess both ``ria" and ``thia" respectively. In these cases, ``ria" is pronounced as ``ree-ea" while ``thia'' is pronounced as ``thi-a". Due to these
cases, there occurs a limitation on being able to diagnose the correct letter substitution for these types of pronunciations.