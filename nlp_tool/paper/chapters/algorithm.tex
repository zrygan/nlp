\section{Algorithm}

The regulating body for the Filipino language of the Philippine government, the Komisyon sa Wikang Filipino (KWF), provides a manual on the orthography of
the language~\cite{KWF_MMP} which includes guidelines on nativization of loan
words.

In particular, the KWF provides a list of substitutions for common letter
combinations and sounds found in English loan words to their Filipino
equivalents. Based on these guidelines, we developed a rule-based algorithm to
nativize English drug names to Filipino.

We enumerate the types of rules required for the nativization algorithm below.

\subsection{Monograph Substitutions}

These are basic letter substitutions where a single letter in English is
replaced with a single letter in Filipino. Since both languages use the Latin
script, many letters have direct equivalents while some being equivalent with
itself. For example, the letter ``a'' (\textbf{a}bstact) in English is pronounced
the same way as the letter ``a'' (\textbf{a}bstrak) in Filipino. While the letter
``v'' (\textbf{v}olleyball) does not exist in Filipino phonetics and is replaced
with the letter ``b'' (\textbf{b}alibol) which has a similar sound. Furthermore,
other letters such as ``x'' (inde\textbf{x}, \textbf{x}-ray) and ``q'' (\textbf{q}ueen) do not
a single letter mapping, they are instead replaced with ``ks'' (inde\textbf{ks},
\textbf{eks}-ray) and ``kuw'' (\textbf{kuw}in).

In the table below, we summarize the common monograph substitutions used in our
nativization algorithm.

\begin{table}[h]
\centering
\begin{tabular}{lll}
\hline
	extbf{English} & \textbf{Filipino} & \textbf{Example} \\
\hline
ph & f   & phone $\rightarrow$ fon \\
v  & b   & volleyball $\rightarrow$ bolibol \\
z  & s   & zipon $\rightarrow$ sipon \\
j  & dy  & jeep $\rightarrow$ dyip \\
\~n & ny & ba\~no $\rightarrow$ banyo \\
q(u) & kuw & queen $\rightarrow$ kuw\'in \\
x  & ks  & index $\rightarrow$ indeks \\
\hline
\end{tabular}
\caption{Common monograph substitutions used in the nativization algorithm.}
\label{tab:mono-rules}
\end{table}

\subsection{Digraphs}

English has several letter combinations or digraphs that represent specific
sounds. 

We again use the KWF manual to identify common digraphs in English and their
Filipino equivalents. For example, the digraph ``ch'' (\textbf{ch}irp) is
mapped to ``ts'' (\textbf{ts}irp) in Filipino. Another example is the digraph
``sh'' (\textbf{sh}ip, squi\textbf{sh}) which is replaced with ``sy'' (\textbf{sy}ip) or ``s'' (\textbf{iskuwis}) in
Filipino. A special diragph is the ``ng'' which is the only digraph that is a
letter in the Filipino alphabet.

Furthermore, English also uses a few vowel digraphs that are mapped to Filipino
equivalents. For example, the digraph ``oo'' (\textbf{oo}ze) is replaced with
``u'' (\textbf{u}z) in Filipino. Another example is the digraph ``ea''
(\textbf{ea}sy) which is mapped to ``i'' (\textbf{i}si) in Filipino.

In the table below, we summarize the common digraph substitutions used in our
nativization algorithm.

\begin{table}[h]
\centering
\begin{tabular}{lll}
\hline
	extbf{Pattern} & \textbf{Filipino} & \textbf{Example} \\
\hline
ch & ts      & chocolate $\rightarrow$ tsokolate \\
sh & sy / s  & ship $\rightarrow$ syip; squish $\rightarrow$ iskuwis \\
ng & ng      & sing $\rightarrow$ sing \\
oo & u       & ooze $\rightarrow$ uz \\
ea & i       & easy $\rightarrow$ isi \\
c + e/i/y & s & centro $\rightarrow$ sentro \\
c elsewhere & k & computer $\rightarrow$ kompyuter \\
th & t       & thiamine $\rightarrow$ tayamin \\
tion / sion & syon & education $\rightarrow$ edukasyon \\
\hline
\end{tabular}
\caption{Common digraph and pattern substitutions used in the nativization algorithm.}
\label{tab:digraph-rules}
\end{table}

\subsection{Vowel Unpredictability}

Lastly, one of the biggest limitations that we faced whilst encountering the
drug names were the various differences in pronunciation involving ``y",
``i", and the diphthong ``ay". Throughout different drug names, there are
multiple pronunciations that may involve these letters. For example, the drug
name ``diphenhydramine" has the first ``i" pronounced with an ``ay" sound,
compared to the second ``i" which has an ``i" sound. These multiple types of
pronunciations make it difficult to standardize a single rule based purely on
graphemes. There are other occurrences in which two drug names such as
``triamcinolone" and ``hydrochlorothiazide" both contain sequences like
``ria" and ``thia" respectively. In these cases, ``ria" may be pronounced as
something close to ``ree-a" while ``thia" may be realized as ``thi-a". Due to
these cases, it is challenging to diagnose the correct letter substitution for
these types of pronunciations using a fixed set of orthographic rules.

In addition to the basic substitutions, the algorithm also handles several
common English suffixes and spelling patterns that occur in drug names. For
instance, words ending in ``-ide", ``-ate", and ``-one" are typically
pronounced as ``ayd", ``eyt", and ``own" respectively, so the algorithm maps
these endings to ``-ayd", ``-eyt", and ``-own" in Filipino spellings.
Similarly, final silent ``e" in English words is usually dropped in the
nativized form, and sequences such as ``ein" are simplified to ``in". These
patterns were incorporated based on recurring forms observed in pharmaceutical
terminology and are implemented alongside the monograph and digraph rules
described above.

The rules described in this section are implemented in the
	exttt{apply\_phonological\_rules} method of the
	exttt{FilipinoCFGParser} class in \texttt{naturalsyon.py}. In addition to
the substitutions summarized in Tables~\ref{tab:mono-rules} and
\ref{tab:digraph-rules}, the implementation includes suffix mappings (e.g.,
	extit{-ate} $\rightarrow$ \textit{-eyt}, \textit{-ide} $\rightarrow$
	extit{-ayd}, \textit{-one} $\rightarrow$ \textit{-own}) and vowel-glide
patterns (e.g., inserting \textit{y} or \textit{w} in sequences such as
	extit{oa}, \textit{ia}, \textit{ua}), which were tuned to better capture
the behavior of pharmaceutical terms.