\section{Algorithm}

% THIS IS CLARENCE'S ORIGINAL WRITE-UP
% INCORPORATE IF NEEDED LATER
% ===========================================
% In order to evaluate the nativization of drug names, we conducted preliminary research on standardized rules by the KWF on English loaned words. 
% Based on these translations, we conducted basic letter substitutions such as replacing ``ph" with ``f", ``z" with ``s", ``th" with ``t", ``ch" with ``k", ``v" with ``b", and 
% "x" with ``ks". Furthermore, there were other letter substitutions such as replacing ``c" with ``s" when followed by ``e", ``y", or ``i", and replacing ``c" with ``k" in other cases.
% This captures the soft and hard ``c" sounds in English. However, there are important limitations to note with the letter ``g". This letter has both a ``soft" and ``hard"
% pronunciation depending on the following vowel, similar to ``c". However, there are multiple special cases in which this algorithm is not followed. One such example
% is the word ``get", which has a hard ``g" sound despite being followed by ``e". Due to these exceptions, there are limitations to drug names that contain the letter ``g" within them.

% Aside from these basic letter substitutions, there are other special exceptions and rules that were applied. For example, any drug name that has ``ee" in it's spelling will be
% substituted with ``i". The same occurrence applies to ``oo", which will be replaced with the letter ``u". However other characters that have duplicated letters in a row are substituted with just 
% one letter instead. Additionally, if a drug name ends with the letter ``e", it is highly noticeable that the ``e" is silent and removed in pronunciation. Therefore, any drug name that ends with ``e" 
% will have the final ``e" removed. Moreoever, there are multiple drug names that end with the following: ``-ide", ``-ate", and ``-one", which are typically pronounced as ``ayd", ``eyt", and ``own" 
% respectively. To account for these common endings, we have replaced the endings accordingly. There are also a considerable amount of other spellings in drug names such as ``ein" that gets 
% substituted with ``in" and ``thi" that may follow with a ``a" and ``o" with ``ti[a/o]" depending on the following letter.

% Lastly, one of the biggest limitations that we faced whilst encountering the drug names were the various difference of pronunciations between ``y", ``i", and ``ay". Throughout different drug names,
% there are multiple pronunciations that may involve these letters. For example, the drug name ``diphenhydramine" has the first ``i" as a ``ay" sound, compared to the second ``i" which has a ``ee" sounnd.
% These multiple different types of of pronunciations made it difficult to standardize a single rule based on graphemes on how to nativize these drug names. There are other occurences in which two words such as 
% ``triamcinolone" and ``hydrochlorothiazide" in which possess both ``ria" and ``thia" respectively. In these cases, ``ria" is pronounced as ``ree-ea" while ``thia'' is pronounced as ``thi-a". Due to these
% cases, there occurs a limitation on being able to diagnose the correct letter substitution for these types of pronunciations.
% ===========================================
% THIS IS CLARENCE'S ORIGINAL WRITE-UP
% INCORPORATE IF NEEDED LATER

The regulating body for the Filipino language of the Philippine government is
the Komisyon sa Wikang Filipino (KWF) provides a manual on the orthography of
the language~\cite{KWF_MMP} which includes guidelines on nativization of loan
words.

In particular, the KWF provides a list of substitutions for common letter
combinations and sounds found in English loan words to their Filipino
equivalents. Based on these guidelines, we developed a rule-based algorithm to
nativize English drug names to Filipino.

We enumerate the types of rules required for the nativization algorithm below.

\subsection{Monograph Substitutions}

These are basic letter substitutions where a single letter in English is
replaced with a single letter in Filipino. Since both languages use the Latin
script, many letters have direct equivalents while some being equivalent with
itself. For example, the letter ``a'' (\textbf{a}bstact) in English is pronounced
the same way as the letter ``a'' (\textbf{a}bstrak) in Filipino. While the letter
``v'' (\textbf{v}olleyball) does not exist in Filipino phonetics and is replaced
with the letter ``b'' (\textbf{b}alibol) which has a similar sound. Furthermore,
other letters such as ``x'' (inde\textbf{x}, \textbf{x}-ray) and ``q'' (\textbf{q}ueen) do not
a single letter mapping, they are instead replaced with ``ks'' (inde\textbf{ks},
\textbf{eks}-ray) and ``kuw'' (\textbf{kuw}in).

In the table below, we summarize the common monograph substitutions used in our
nativization algorithm.

\begin{table}[h]
\centering
\caption{Monograph Phoneme Mappings}
\begin{tabular}{|c|c|c|c|}
\hline
\textbf{Source} & \textbf{Target} & \textbf{Source} & \textbf{Target} \\
\hline
b & b & p & p \\
k & k & r & r \\
d & d & s & s \\
g & g & t & t \\
h & h & w & w \\
l & l & y & y \\
m & m & v & b \\
n & n & f & p \\
\hline
j & dy & z & s \\
q & k & x & ks \\
\hline
\end{tabular}
\end{table}

\subsection{Digraphs}

English has several letter combinations or digraphs that represent specific
sounds. 

We again use the KWF manual to identify common digraphs in English and their
Filipino equivalents. For example, the digraph ``ch'' (\textbf{ch}irp) is
mapped to ``ts'' (\textbf{ts}irp) in Filipino. Another example is the digraph
``sh'' (\textbf{sh}ip, squi\textbf{sh}) which is replaced with ``sy'' (\textbf{sy}ip) or ``s'' (\textbf{iskuwis}) in
Filipino. A special diragph is the ``ng'' which is the only digraph that is a
letter in the Filipino alphabet.

Furthermore, English also uses a few vowel digraphs that are mapped to Filipino
equivalents. For example, the digraph ``oo'' (\textbf{oo}ze) is replaced with
``u'' (\textbf{u}z) in Filipino. Another example is the digraph ``ea''
(\textbf{ea}sy) which is mapped to ``i'' (\textbf{i}si) in Filipino.

In the table below, we summarize the common digraph substitutions used in our
nativization algorithm.


\begin{table}[h]
\centering
\caption{Digraph Phoneme Mappings}
\begin{tabular}{|c|c|}
\hline
\textbf{Source} & \textbf{Target} \\
\hline
ph & p \\
th & t \\
ch & ts \\
sh & sy \\
ng & ng \\
ny & ny \\
qu & kw \\
ñ & ny \\
\hline
\end{tabular}
\end{table}

\subsection{Vowel Unpredictability}

Lastly, one of the biggest limitations that we faced whilst encountering the
drug names were the various difference of pronunciations between ``y", ``i",
and ``ay". Throughout different drug names, there are multiple pronunciations
that may involve these letters. For example, the drug name ``diphenhydramine"
has the first ``i" as a ``ay" sound, compared to the second ``i" which has a
``ee" sounnd. These multiple different types of of pronunciations made it
difficult to standardize a single rule based on graphemes on how to nativize
these drug names. There are other occurences in which two words such as
``triamcinolone" and ``hydrochlorothiazide" in which possess both ``ria" and
``thia" respectively. In these cases, ``ria" is pronounced as ``ree-ea" while
``thia'' is pronounced as ``thi-a". Due to these cases, there occurs a
limitation on being able to diagnose the correct letter substitution for these
types of pronunciations.