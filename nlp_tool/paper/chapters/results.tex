\section{Results and Discussion}

We evaluated the algorithm on the three datasets described in the experimental
setup: 199 generic drug names, 262 drug brand names, and 100 common English
words used in Filipino. The performance was measured as accuracy, computed as
the proportion of words for which the algorithm's output exactly matched the
manually curated nativized form.

On the generic drug names, the system achieved an accuracy of 88.94\%,
correctly producing nativized forms such as \textit{acetaminophen}
$\rightarrow$ \textit{asetaminofen} and \textit{azithromycin} $\rightarrow$
	extit{asitromaysin}. Accuracy decreased to 67.18\% on brand names and 36\%
on common English words. This pattern reflects the greater orthographic and
phonological variability of proprietary names and informal borrowings compared
with standardized International Nonproprietary Names.

Many of the remaining errors involve the treatment of ``y'', ``i'', and the
diphthong ``ay'', particularly in cases where multiple plausible Filipino
realizations exist. Words containing sequences such as ``ria'' or ``thia" may
be pronounced and nativized in more than one way, and the current set of
rules does not capture all of these possibilities. These observations
underscore the difficulty of modeling fine-grained pronunciation and
orthographic patterns through fixed rules alone and motivate future work that
incorporates phonetic information or data-driven methods to improve coverage
on brand names and common words.