\section{Results and Discussion}

In testing the  effectiveness of this nativization algorithm, we utilized multiple datasets containing drug names, drug brand 
names, and regular words. The performance of the algorithm was evaluated based on its ability to accurately nativize drug names
into Filipino. We calculated the accuracy of the nativization by comparing the algorithm's output against a manually curated 
list of nativized words. The goal of this evaluation was to determine how well the algorithm could generalize across different 
datasets whether it be for drug names, brand names, or common words. The results indicated that the algorithm performed well 
on the initial dataset, achieving an accuracy of 88.94\%. However, when tested on additional datasets, the accuracy varied, with 
a drop to 67.18\% on brand names and 36\% on regular words. This suggests that while the algorithm is effective for drug names, 
it may require further refinement to handle the nuances of brand names and common words. Overall, it achieved its desired result 
in capturing the nativized forms of drug names in Filipino, but further work is needed to enhance its versatility across different
types of words.

The most significant limitation of this project is the persistent dilemma involving the correct use of “y,” “ay,” and “i,” as the 
algorithm produces forms based on observed rules and patterns found within the dataset. These inconsistencies underscore the 
difficulty of fully capturing nuanced pronunciation and orthographic patterns through rule-based methods alone. Nevertheless, 
while the algorithm successfully meets its primary objective of nativizing drug names into Filipino, the challenges identified
particularly in handling vowel-glide variations—highlight opportunities for further refinement. Future iterations of the tool 
may benefit from incorporating more advanced approaches, such as phonetic modeling or machine learning techniques, to enhance 
both the accuracy and adaptability of the nativization process across a broader range of word types.