\section{Geographic Determinants}

This section employs linguistic cartography and a focused literature review to
examine how geography has shaped the distribution and divergence of Philippine
languages. Linguistic cartography—mapping language data onto spatial terrain—enables
the visualization of geolinguistic variation, while literature contextualizes these
patterns through historical and sociocultural perspectives. Together, these methods
offer complementary insight into how environmental constraints have contributed to
linguistic diversification across the archipelago.

The Philippines is a topographically complex nation, composed of over 7,600 islands
grouped into \emph{Luzon}, \emph{Visayas}, and \emph{Mindanao}. Each region
possesses distinct ecological and historical conditions that correspond to linguistic
variation~\cite{Boquet_2017}. Scholars attribute this diversity largely to the
archipelago's fragmented geography, which has historically restricted mobility and
sustained linguistic isolation~\cite{Blust_1991}. Mountains, rivers, and ocean
channels function both as connectors and separators, influencing how language
communities form, expand, and maintain distinct identities.

Historically, the smallest socio-political unit of pre-colonial society—the
\emph{barangay}—was typically situated near water sources such as rivers or coastal
areas. These locations facilitated trade and communication, yet simultaneously
created micro-environments of linguistic autonomy. The archipelagic layout of the
country further intensified this phenomenon: neighboring islands could remain
mutually unintelligible despite geographic proximity. Over time, these conditions
produced a mosaic of related but distinct linguistic systems.

Blust's Austronesian dispersal hypothesis~\cite{Blust_1991} suggests that a single
protolanguage, introduced by early settlers from the Asian mainland, gradually
diverged through localized adaptation and restricted contact. As communities became
geographically isolated, subtle phonological and lexical innovations accumulated,
leading to the 175 documented Philippine languages, two of which are now
extinct~\cite{ethnologue2025}. This section highlights a representative subset of
these languages—ranging from dominant regional lingua francas to endangered local
tongues—shown in \cref{tab:ph_languages}.

These languages collectively represent the major linguistic regions of the
Philippines. Tagalog, Kapampangan, and Pangasinan
dominate Central Luzon's plains; Ilocano and
Paranan reflect the mountainous north and isolated Sierra Madre
communities. In the Visayas, Cebuano, Hiligaynon,
Kinaray-a, and Waray form a contiguous yet
diversified cluster of Bisayan languages separated by narrow sea channels.
Meanwhile, Bikol, Romblomanon, and Masbatenyo
occupy transitional zones between Luzon and the Visayas, serving as
intermediate varieties shaped by frequent inter-island contact. Mindanao's
linguistic landscape, represented by Tausug and Chavacano
, displays intense multilingual interaction and creolization due to
historic trade networks. Yami, spoken in the Batanes Islands, marks
the northern Austronesian periphery, maintaining archaic features shared with
Formosan languages.

\subsection{Methodology}

We employed two official linguistic atlases published by the Komisyon ng Wikang
Filipino (KWF): the \emph{Mapa ng mga Wika} and the \emph{Mapa ng mga Wika ng
Katutubong Pamayanang Kultural}. These maps delineate the spatial distribution of
languages across Philippine administrative boundaries and identify overlapping
zones of multilingual interaction~\cite{Marrion2025MapaKultural,
KWF2025MapaPilipinas}. To enhance precision, we generated a custom digital map
using boundary data from the United Nations Office for the Coordination of
Humanitarian Affairs (OCHA) and rendered it with
Matplotlib\footnote{\url{https://matplotlib.org/}} and
GeoPandas\footnote{\url{https://geopandas.org/en/stable/}} in
Python~\cite{OCHA_Philippines_2018_PHL_Admin_Boundaries, Hunter:2007,
kelsey_jordahl_2020_3946761}. This process ensured spatial consistency between
linguistic and geographic datasets.

Topographic data were obtained from the National Mapping and Resource Information
Authority (NAMRIA)~\cite{NAMRIA_Downloads}, including elevation models, hydrology,
and terrain classifications. By overlaying linguistic boundaries on these physical
features, we identified correlations between geography and language distribution.
In particular, we focused on how elevation and water systems delineate the
boundaries of the languages in \cref{tab:ph_languages}.

The analysis followed three phases: (1) geospatial alignment of linguistic and
physical layers; (2) detection of convergence between linguistic boundaries and
terrain discontinuities; and (3) qualitative interpretation through historical and
sociolinguistic literature. This mixed-method approach integrates computational
mapping with cultural geography, facilitating both quantitative and interpretive
insights.

\subsection{Results and Discussion}

The generated linguistic map, presented in \cref{fig:lingmap_out}, reveals clear
patterns of correspondence between physical terrain and language boundaries.
Ilocano and Paranan, for instance, are divided by the Cordillera and Sierra Madre
ranges, while the seas separating Panay, Negros, and Leyte correspond precisely to
the divisions among Hiligaynon, Kinaray-a, and Waray. The case of Bikol,
Masbatenyo, and Romblomanon illustrates an intermediate linguistic zone where
mutual intelligibility correlates with ease of maritime travel.

In contrast, lowland regions such as Central Luzon and Northern Mindanao show
diffuse linguistic borders, reflecting trade, migration, and urbanization. The
presence of Chavacano in Zamboanga exemplifies contact-induced
change, where Spanish influence merged with Cebuano and Tausug substrates to form
a stable creole. Similarly, Yami in the Batanes archipelago
preserves linguistic continuity with Taiwan's indigenous languages due to
geographic proximity and historical seafaring ties.

Overall, the spatial correlations underscore geography as both a limiting and
enabling force: rugged terrains and isolated islands encourage linguistic
diversification, while navigable waterways foster linguistic contact and hybrid
formation. These findings align with global typologies of geography-driven
diversity, placing the Philippines alongside regions such as Indonesia and Papua
New Guinea in the broader Austronesian context.

\begin{figure}[h]
    \centering
    \includegraphics[width=\columnwidth]{lingmap/artifacts/lingmap_out.png}
    \caption{Generated linguistic map showing the spatial distribution of selected
    Philippine languages overlaid on topographic contours.}
    \label{fig:lingmap_out}
\end{figure}

The mapping code and datasets are publicly available in the
\href{https://github.com/zrygan/nlp/tree/master/language_similarity/lingmap}
{\texttt{language\_similarity/lingmap}} directory for transparency and
reproducibility. The results affirm that the Philippine archipelago's natural
terrain and hydrology are central to understanding the historical pathways and
current distribution of its languages.
