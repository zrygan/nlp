\section{Geographic Determinants}

This section will use linguistic cartography and a literature review as its
primary methodology. \todo (IS THIS EVEN GOOD? REVISE)

\todo (ADD MORE WORK HERE)

The Philippines is a topographically complex (with mountainous and archipelagic
country), it is composed of three main island groups \emph{Luzon},
\emph{Visayas}, and \emph{Mindanao}. Each with their own geolinguistic
variation~\cite{Boquet_2017}. Literature attributes the Philippines's
geolinguistic diversity to its geographic features serving as a separation
between cultural or linguistic groups~\cite{Blust_1991}.

Historically, the smallest socio-political unit of Philippine society was the
\emph{barangay}, whose pre-colonial location was often determined by proximity
to water — rivers, coasts, and estuaries. Given the archipelagic and
mountainous nature of the Philippines, settlement distribution may have
constrained inter-barangay communication, influencing the evolution and
divergence of local languages. \todo (no citation)

\subsection{Analysis: Linguistic Cartography}

\subsubsection{Mapping the Selected Philippine Languages}

Python code was written using
Matplotlib\footnote{\url{https://matplotlib.org/}} and
GeoPandas\footnote{\url{https://geopandas.org/en/stable/}} to generate a
colored map of the Philippines where each of the selected Philippine languages
are highlighted, as shown in~\cref{fig:lingmap_out}. The corresponding code may
is available in the
\href{https://github.com/zrygan/nlp/tree/master/language_similarity/lingmap}{\texttt{language\_similarity/lingmap}}
directory of the GitHub repository.

\begin{figure}[h]
    \centering
    \includegraphics[width=\columnwidth]{lingmap/artifacts/lingmap_out.png}
    \caption{Generated linguistic map output.}\label{fig:lingmap_out}
\end{figure}

