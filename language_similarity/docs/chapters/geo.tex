\section{Geographic Determinants}

This section will use linguistic cartography and a literature review as its
primary methodology. \todo (IS THIS EVEN GOOD? REVISE)

\todo (ADD MORE WORK HERE)

The Philippines is a topographically complex (with mountainous and archipelagic
country), it is composed of three main island groups \emph{Luzon},
\emph{Visayas}, and \emph{Mindanao}. Each with their own geolinguistic
variation~\cite{Boquet_2017}. Literature attributes the Philippines's
geolinguistic diversity to its geographic features serving as a separation
between cultural or linguistic groups~\cite{Blust_1991}.

Historically, the smallest socio-political unit of Philippine society was the
\emph{barangay}, whose pre-colonial location was often determined by proximity
to water---rivers, coasts, and estuaries. Given the archipelagic and
mountainous nature of the Philippines, settlement distribution may have
constrained inter-barangay communication, influencing the evolution and
divergence of local languages. \todo (no citation)

We will do a comparison of the selected languages by comparing geographic determinants.

\subsection{Proving Blust's 1991 Hypothesis by Linguistic Cartography}

Blust's 1991 hypothesis in~\cite{Blust_1991} mentions that Philippine languages started
from a protolanguage which were used by the mainland pre-colonial Filipino people who then
spread out to numerous coastal areas and islands. And due to the geographic features of the
Philippines, he highlights that small linguistic variation and borrowing over long periods
of time lead to a massive shift in the language use of each social unit which.
As a consequence of this, the protolanguage grew to the 175 languages of the Philippines, two
of which are now extinct~\cite{ethnologue2025}.

We will show that Blust's 1991 hypothesis is indeed true through topographic and linguistic cartography.

\subsection{Linguistic Atlas}

Python code was written using
Matplotlib\footnote{\url{https://matplotlib.org/}} and
GeoPandas\footnote{\url{https://geopandas.org/en/stable/}} to generate a
colored map of the Philippines where each of the selected Philippine languages
are highlighted, as shown in~\cref{fig:lingmap_out}. The corresponding code is
available in the
\href{https://github.com/zrygan/nlp/tree/master/language_similarity/lingmap}{\texttt{language\_similarity/lingmap}}
directory of the GitHub repository.

\begin{figure}[h]
    \centering
    \includegraphics[width=\columnwidth]{lingmap/artifacts/lingmap_out.png}
    \caption{Generated linguistic map output.}\label{fig:lingmap_out}
\end{figure}

We then cross-referenced the generated linguistic atlas using the Komisyon ng
Wikang Filipino's (KWF) \emph{Mapa ng mga Wika} or linguistic atlas.
