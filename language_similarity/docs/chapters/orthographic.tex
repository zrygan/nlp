\section{Orthographic Similarity}

The orthographic similarity of languages refers to the spelling convention of a certain language. This metric is 
important because it can give insights into the historical and cultural relationships between languages. Languages 
that share similar orthographic systems may have been influenced by the same writing systems or may have had 
significant contact in the past. This gives us valuable insights as to how languages have evolved over time and how they
are related to one another.

\subsection{Methodology}

To compute the orthographic similarity between the selected Philippine languages, we used the Jaccard similarity
coefficient. The Jaccard similarity coefficient is a statistical measure used to compare the similarity and diversity of
sample sets. It is defined as the size of the intersection divided by the size of the union of the sample sets. In our case,
we defined the sample sets as the unique characters used in the orthography of each language. The Jaccard similarity coefficient is computed using the formula:
\begin{equation}
    J(A, B) = \frac{|A \cap B|}{|A \cup B|}
\end{equation}
where \(A\) and \(B\) are the sets of unique trigrams in the orthography of two languages.
We extracted the unique characters from the corpora of each language and computed the Jaccard similarity coefficient for each pair of languages. We stored these values
in a similarity matrix, which is presented in \cref{fig:ortho_sim_matrix}.
\begin{figure}[h]
    \centering
    \includegraphics[width=\columnwidth]{chapters/artifacts/orthographic_heat_map.png}
    \caption{Orthographic Similarity Matrix using Jaccard Similarity Coefficient}
    \label{fig:ortho_sim_matrix}
\end{figure}

\subsection{Results and Discussion}

The orthographic similarity matrix in \cref{fig:ortho_sim_matrix} reveals interesting patterns in how Philippine 
languages relate through their written forms. The overall similarity values range from moderate to high, with considerable 
variation across language pairs. Several languages that are geographically close or linguistically related show relatively 
weak orthographic connections, while others maintain strong similarity. This variability tells us something important: 
the adoption of Latin script across Philippine languages has followed different paths, shaped by distinct standardization 
processes, missionary influences, and local literacy traditions.

The Bisayan languages form a clear cohesive cluster in the orthographic data. Hiligaynon and Cebuano achieve a strong 
similarity score of 0.9391, reflecting their close linguistic relationship as Central Bisayan languages. What's particularly 
interesting is that Romblomanon shows even stronger orthographic connections—scoring 0.9242 with Hiligaynon and 0.8870 
with Cebuano. Kinaray-a maintains solid ties to both Hiligaynon (0.9067) and Cebuano (0.8805), confirming the orthographic 
similarity of Western Visayan languages. Waray-Waray shows somewhat more moderate connections to the other Bisayan languages, 
though still maintaining respectable similarity scores. These patterns suggest that spelling conventions within the Bisayan 
group have remained relatively consistent, likely due to shared educational and religious materials during Spanish 
colonization.

Romblomanon emerges as a particularly interesting bridge language in the orthographic data. It achieves the highest 
similarity with Hiligaynon (0.9242) and shows strong connections to both Tagalog (0.8666) and Cebuano (0.8870). The 
Romblomanon-Kinaray-a pair scores an impressive 0.9143, suggesting that these two languages from neighboring island 
groups have maintained very similar spelling conventions. This makes sense given Romblon's position between major 
linguistic zones—the writing systems likely influenced each other through trade and administrative contact during 
both the Spanish colonial period and modern times.

Bikol and Masbatenyo present another case of exceptionally high orthographic similarity at 0.9206, their highest 
mutual score. This reflects their geographic proximity in the Bicol Peninsula and adjacent islands. These languages 
have clearly maintained similar writing conventions, likely due to shared educational and religious materials during 
Spanish colonization and continuing through modern standardization efforts. Their similarity suggests that the Bicol region had the same literary heritage.

Tagalog shows interesting orthographic patterns that reflect its status as a major regional language. It pairs well 
with Cebuano (0.8877) and Hiligaynon (0.8811), and also maintains strong similarity with Bikol (0.8298) and Romblomanon 
(0.8666). However, its connection to Kapampangan (0.7985) is more moderate, suggesting that despite being geographic 
neighbors in Central Luzon, their orthographic conventions have developed with some independence. The relatively high 
Tagalog-Cebuano score is notable—these are the two most widely spoken Philippine languages, and their similar 
orthographies likely reflect mutual influence through education and media.

Chavacano stands out as the most orthographically divergent language in the matrix. Its highest scores 
are with Pangasinan (0.5228) and Paranan (0.4960), with most other languages scoring between 0.35-0.45. This pattern 
reflects Chavacano's unique status as a Spanish-based creole with a spelling system that differs fundamentally from 
Austronesian languages. Chavacano uses different letter combinations and orthographic conventions inherited from Spanish, 
creating a written form that looks distinctly different from its geographic neighbors. However, the fact that it shows 
any orthographic similarity at all suggests that the standardized Latin script provides some common ground, even for 
a creole language with such different origins.

Ilocano's orthographic profile is quite distinct from most Philippine languages. Its scores are consistently in the 
moderate-to-low range, with its highest similarities being with Paranan (0.7235) and Pangasinan (0.6818). Ilocano's 
orthographic similarity with central Philippine languages—Hiligaynon (0.5583), Cebuano (0.5588), and Tagalog 
(0.5669)—is relatively low, suggesting that Northern Luzon languages have developed distinct spelling conventions. 
This likely reflects different influences during Spanish colonization, with Northern Luzon potentially 
having different standardization traditions than the more heavily populated central regions.

Paranan presents an interesting orthographic profile. It achieves its highest similarities with Pangasinan (0.7828), 
Ilocano (0.7235), and Yami (0.7103), suggesting some Northern Luzon orthographic commonalities. However, these scores 
are moderate rather than exceptionally high, indicating that even within this geographic region, each language has 
maintained somewhat independent spelling conventions. Paranan's isolated position in the Sierra Madre mountains may 
have contributed to its orthographic development following a somewhat different path from its neighbors.

Yami presents an intriguing case. Its similarities are moderate across the board, with the highest connections to 
Pangasinan (0.7769), Bikol (0.7047), and Paranan (0.7103). These moderate scores might reflect Yami's geographic 
isolation in the Batanes Islands—the language has been written down less frequently and may have been subject to 
different orthographic influences. Its position near Taiwan could mean connections to Formosan language writing 
systems, or simply that it underwent less intensive standardization than more widely spoken languages.

Kapampangan shows a somewhat isolated orthographic profile. Its highest similarity is with Tagalog (0.7985), which 
makes sense given their geographic proximity in Central Luzon and likely shared educational systems. However, 
Kapampangan scores lower with most other languages—including 0.7033 with Cebuano and only 0.5421 with Ilocano. 
The orthographic distinctiveness suggests that Kapampangan has maintained unique spelling conventions, possibly 
reflecting its literary heritage in Pampanga province. The language 
has a long history of written materials, which may have allowed it to develop and preserve distinct orthographic 
features.

Pangasinan emerges as having relatively balanced orthographic similarities across multiple languages. It scores 
well with Paranan (0.7828), Bikol (0.7926), Masbatenyo (0.7906), and Yami (0.7769). This pattern positions Pangasinan 
as something of a linguistic crossroads in orthographic terms, showing moderate similarity to both northern and 
southern varieties. The consistency of these mid-to-high scores suggests that Pangasinan's writing system incorporates 
features that are broadly representative of Philippine orthographic conventions, making it somewhat of a "middle ground" 
among Philippine spelling systems.

Tausug's orthographic scores are moderate, ranging mostly from 0.59-0.77. Its strongest connections are with Bikol 
(0.7762), Masbatenyo (0.7464), and Romblomanon (0.7217). This moderate level of similarity is interesting given 
Tausug's position in the far south of the archipelago. The language has been influenced by different educational 
systems and has historical connections to Arabic script traditions in the Muslim areas of Mindanao, even though 
it's now written in Latin script. These influences may have shaped its orthographic conventions in ways that differ 
from languages in the central and northern Philippines.

When we compare these orthographic patterns with the geographic distribution in \cref{fig:lingmap_out}, we see 
that geography provides only a partial explanation for orthographic similarity. The Bisayan cluster maintains cohesion, 
and Northern Luzon languages show some clustering tendencies, but the correlation is not absolute. This makes sense: 
orthography is highly susceptible to standardization efforts, missionary influences, and educational policies that 
can cut across geographic boundaries. Languages separated by water or mountains might still develop similar spelling 
systems if they were served by the same missionary groups or educational institutions.

The orthographic data reveals which language communities have had closer contact in written contexts—through shared 
educational materials, religious texts, or administrative documents. The high Romblomanon-Hiligaynon and Bikol-Masbatenyo 
scores suggest said shared heritage. The lower scores for Chavacano and the variable patterns for isolated 
languages like Yami and Paranan show how geographic and cultural isolation can lead to their orthography developing independently.

These findings show that orthographic similarity tells a specific story about the history of literacy. The matrix shows us that while Philippine languages have converged on using the Latin script, 
they've done so in ways that reflect their unique histories and the particular contexts in which literacy was introduced 
and standardized. The variation we see in orthographic similarity—from Chavacano's dramatic divergence to the tight 
clustering of Bisayan languages—reflects centuries of decisions by educators, linguists, missionaries, and communities 
about how to represent their speech in writing. Future research examining specific orthographic features—like diacritic 
usage, digraph conventions, and representation of sounds that do not exist in Spanish—could provide even more insight 
into how these writing systems developed and why some languages maintain more similar conventions than others.