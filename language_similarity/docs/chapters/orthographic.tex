\section{Orthographic Similarity}

The orthographic similarity of languages refers to the spelling convention of a certain language. This metric is 
important because it can give insights into the historical and cultural relationships between languages. Languages 
that share similar orthographic systems may have been influenced by the same writing systems or may have had 
significant contact in the past. This gives us valuable insights as to how languages have evolved over time and how they
are related to one another.

\subsection{Methodology}

To compute the orthographic similarity between the selected Philippine languages, we used the Jaccard similarity
coefficient. The Jaccard similarity coefficient is a statistical measure used to compare the similarity and diversity of
sample sets. It is defined as the size of the intersection divided by the size of the union of the sample sets. In our case,
we defined the sample sets as the unique characters used in the orthography of each language. The Jaccard similarity coefficient is computed using the formula:
\begin{equation}
    J(A, B) = \frac{|A \cap B|}{|A \cup B|}
\end{equation}
where \(A\) and \(B\) are the sets of unique trigrams in the orthography of two languages.
We extracted the unique characters from the corpora of each language and computed the Jaccard similarity coefficient for each pair of languages. We stored these values
in a similarity matrix, which is presented in \cref{fig:ortho_sim_matrix}.
\begin{figure}[h]
    \centering
    \includegraphics[width=\columnwidth]{artifacts/orthographic_heat_map.png}
    \caption{Orthographic Similarity Matrix using Jaccard Similarity Coefficient}
    \label{fig:ortho_sim_matrix}
\end{figure}

\subsection{Results and Discussion}

